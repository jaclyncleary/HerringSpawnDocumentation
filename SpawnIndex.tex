%%%%% Preamble %%%%%

% Set document style and font size
\documentclass[12pt]{article}

% Custom DFO Technical Reports
\usepackage{techreport}  

% New definitions: title, year, number, authors, etc
% Note: put words in math mode to avoid case changes (i.e., species names)
\newcommand{\trTitle}{Calculating spawn indices for Pacific herring ($Clupea~pallasii$) in British Columbia, Canada}
\newcommand{\trYear}{2017}
\newcommand{\trReportNum}{XXXX}
\newcommand{\trAuthALong}{Matthew H.~Grinnell}
\newcommand{\trAuthBLong}{Other Authors}  % Jaclyn S.~Cleary, Matt Thompson
\newcommand{\trAuthABack}{Grinnell, M.\,H.}
\newcommand{\trAuthBBack}{Authors, O.}  % Cleary, J.\,S.
\newcommand{\trAuthAEmail}{\texttt{Matthew.Grinnell@dfo-mpo.gc.ca}}
\newcommand{\trAuthBEmail}{\texttt{Other.Authors@dfo-mpo.gc.ca}}  % Jaclyn.Cleary
\newcommand{\trAuthAPhone}{(250)~756.7055}
\newcommand{\trAuthBPhone}{(250)~765.XXXX}  % 7321
%\newcommand{\thisTitle}{This is a test title clupea pallasii}

% New definition: Address
\newcommand{\trAddy}{Fisheries and Oceans Canada\\
Science Branch, Pacific Region\\
Pacific Biological Station\\
3190 Hammond Bay Road\\
Nanaimo, BC \enskip V9T 6N7}

% New definition: build the citation
\newcommand{\trNumPagesRoman}{\pageref{TRlastRoman}}
\newcommand{\trNumPagesArabic}{\pageref{LastPage}}
\newcommand{\trReference}{\trAuthABack{} and \trAuthBBack{} \trYear{}. \trTitle{}. Can. Tech. Rep. Fish. Aquat. Sci. \trReportNum{}: \trNumPagesRoman{} + \trNumPagesArabic{} p.}

% Set document options
%\linenumbers  % For drafts
%\onehalfspacing  % For drafts

% Let it begin
\begin{document}

%%%% Front matter %%%%%

% Format the first few pages
\thispagestyle{fancyplain}
\noindent
\begin{flushleft}
\LARGE
\textbf{\trTitle{}}
\vfill
\Large
\trAuthALong{}, and \trAuthBLong{}
\vfill
\trAddy{}
\vfill
\trYear{}
\vfill
\LARGE
\textbf{Canadian Technical Report of\\
Fisheries and Aquatic Sciences \trReportNum{}}
\lfoot{\includegraphics[height=7mm]{Document/LogoDFO.png}}
\cfoot{}
\rfoot{\includegraphics[height=7mm]{Document/LogoCanada.jpg}}
\end{flushleft}
\clearpage
  % Cover page
\input{Document/TechReportPage.tex}  % Tech report page
\normalsize
\pagenumbering{roman}
\thispagestyle{empty}
\noindent
\begin{center}
Canadian Technical Report of\\
Fisheries and Aquatic Sciences \trReportNum{}
\vfill
\trYear{}
\vfill
\MakeTextUppercase{\trTitle{}}
\vfill
by
\vfill
\trAuthALong{},\footnote{\trAuthADetails{}} and \trAuthBLong{}\footnote{\trAuthBDetails{}}
\vfill
\trAddy{}
\end{center}
\clearpage
  % Inside cover page
\input{Document/ColophonPage.tex}  % Colophon page
\tableofcontents \clearpage  % Table of contents page
\noindent
\section*{ABSTRACT}\addcontentsline{toc}{section}{ABSTRACT}
\trReference{}
\bigskip
This report documents the calculations used to convert spawn survey observations (e.g., number of egg layers, extent of spawn) to the spawn index for Pacific herring ($Clupea$ $pallasii$) in British Columbia (BC), Canada.
There are three types of spawn survey observations: surface observations, underwater observations of spawn on Macrocystis, and underwater observations of spawn on understory.
Data from these three survey types are combined to develop an overall annual spawn index for each stock assessment region in BC.
The spawn index is one component of Pacific herring stock assessments in BC.
Note that the 'spawn index' is not scaled by the spawn survey scaling parameter, $q$ and therefore is not an estimate of spawning stock biomass.
\section*{R\'{E}SUM\'{E}}\addcontentsline{toc}{section}{R\'{E}SUM\'{E}}
\trReference{}
\bigskip
Ceci est la r�sum�...
\vfill
\label{TRlastRoman}
\clearpage
  % Abstact and resume page

% Settings for the main document
\pagenumbering{arabic}  % Regular page numbers
\thispagestyle{empty}  % No page number on first page

%%%%% Main document %%%%%

\section{INTRODUCTION}

This report describes the calculations used to convert spawn survey observations (e.g., number of egg layers, extent of spawn) to spawn indices for Pacific herring (\!\emph{Clupea pallasii}) in British Columbia (BC), Canada.
Note that the `spawn index' is not scaled by the spawn survey scaling parameter, $q$ \citeyearpar[CSAS][]{CSAS2015b}.
Therefore, spawn indices must be scaled by $q$ before being considered to be an estimates spawning stock biomass.
Spawn indices are one component of BC herring stock assessments.

Data for spawn indices are collected by spawn surveys.
There are three types of spawn survey observations: surface survey observations, underwater observations of spawn on Macrocystis, and underwater observations of spawn on understory.
Understory observations have two components: spawn on the bottom, and spawn on vegetation (i.e., algae).
Surface spawn surveys have the coarsest spatial resolution and greatest spatial extent of the three survey types (Ref?).
Surface spawn surveys were the only survey type prior to 1988, and they are still used extensively for minor spawns because they are adequate for narrow spawns in shallow water (Ref?).
Macrocystis and understory spawn surveys are conducted using SCUBA gear.
Underwater SCUBA surveys have been used for all major spawns since 1988, and are thought to be more accurate than surface spawn surveys (Ref?).
Herring spawn surveys began in 1928, but are considered incomplete prior to 1937 because many potential areas were not surveyed (Ref?).

Herring spawn survey observations have a nested hierarchical structure: quadrats are nested within transects, transects are nested within spawns, and spawns are nested within locations.
For stock assessment purposes, locations are nested within sections, sections are nested within statistical areas, and statistical areas are nested within regions.
There are five major and two minor stock assessment regions (SARs) in BC (Figure \ref{figBC}).
Another level of data structure is `beds', which are habitat features as opposed to distinct spatial areas.
Bed widths are used to calculate the spawn index for surface surveys.

\begin{figure}
\centering
\includegraphics[width=\linewidth]{Figures/BC.pdf}
\caption{Boundaries for the British Columbia Pacific herring stock assessment regions (SARs): there are five major SARs (Haida Gwaii, HG; Prince Rupert District, PRD; Central Coast, CC; Strait of Georgia, SoG; and West Coast Vancouver Island, WCVI), and two minor SARs (Area 27, A27; and Area 2 West, A2W).}
\label{figBC}
\end{figure}

The spawn index is calculated by the \textbf{R} \citeyearpar[RCT][]{R-3.3.2} script \texttt{Spawn.R}, which pulls data from the spawn database (Ref?) and does the calculations described in the following sections.
Note that variable names in this document correspond to variable names in the \texttt{Spawn.R} script.

\section{SURFACE SPAWN}

Surface spawn surveys collect data along transects, and calculations are done at the transect, and spawn/bed level.

\subsection{TRANSECT LEVEL CALCULATIONS}

For each bottom type, egg layers is
\begin{equation}
EggLyrs = Layers \times Proportion
\label{eqEggLayersSurf}
\end{equation}
where $Layers$ is the number of egg layers on a given bottom type, and $Proportion$ is the proportion of the transect covered by the bottom type.
At the transect level, the sum of $EggLyrs$ is $EggLyrsTotT$.
That is to say, $EggLyrsTotT$ is the sum of $EggLyrs$ for all the bottom types within a given transect.
For the time period when `Intensity' categories were recorded instead of estimating the number of egg layers, intensity is converted to $EggLyrsTotT$ (Table \ref{tabIntensity}).
Surface egg density is \citep{SchweigertEtal1997}
\begin{equation}
EggDensT = EggLyrsTotT \times 212.218 + 14.698
\label{eqEggDensSurf}
\end{equation}
where $EggDensT$ is in $\text{Eggs} \cdot \text{m}^{-2}$. 

\begin{table}
\centering
\caption{Spawn intensity categories and associated egg layers for surface surveys for the periods 1928--1950, and 1951--1978 (Ref?).}
\begin{tabular}{ccr}
\toprule
\multicolumn{2}{c}{Intensity category} & \\
1928--1950 & 1951--1978 & Egg layers\\
\midrule
0 & 0 & 0.0000 \\
1 & 1 & 0.5529 \\
 & 2 & 0.9444 \\
2 & 3 & 1.3360 \\
 & 4 & 2.1496 \\
3 & 5 & 2.9633 \\
 & 6 & 4.1318 \\
4 & 7 & 5.3002 \\
 & 8 & 6.5647 \\
5 & 9 & 7.8291 \\
\bottomrule
\end{tabular}
\label{tabIntensity}
\end{table}

\subsection{SPAWN/BED LEVEL CALCULATIONS}

At the spawn/bed level, the mean of $EggDensT$ is $EggDensMeanS$.
Two other metrics are required at the spawn/bed level: the spawn/bed length $Length$ and width $WidthS$, both in metres.%
\footnote{$WidthS$ is set to the first non-missing value from the bed width, section width, region width, or observed width (in that order).}
The surface spawn index is
\begin{equation}
SurfSSB = \frac{EggDensMeanS \times Length \times WidthS} {100\,000}
\label{eqBiomassSurf}
\end{equation}
where $SurfSSB$ is in (units?), based on fecundity estimates by \citet{Hay1985}: 200 eggs per gram of female herring, and assuming that females account for 50\% of spawners.

\section{MACROCYSTIS SPAWN}

Macrocystis spawn surveys collect data for individual plants, and calculations are done at the transect, and spawn levels.

\subsection{TRANSECT LEVEL CALCULATIONS}

Several metrics are collected at the transect level: transect length $LengthT$, transect width $WidthT$, and Macrocystis length $LengthMacroT$, all in metres, as well as transect area $AreaT$, in square metres.%
\footnote{$LengthT$ is the width from shore (i.e., transect length), $WidthT$ is the transect width (i.e., $2\,\text{m}$), and $LengthMacroT$ is the length of the Macrocystis bed.
$LengthT$ and $LengthMacroT$ will be the same if there is no recorded Macrocystis bed length.}
In addition, metrics are calculated for mature Macrocystis plants: mean height $HeightMeanT$ in metres, mean egg layers $LayersMeanT$, total number of fronds $FrondsTotT$, and total number of plants $PlantsTotT$.

\subsection{SPAWN LEVEL CALCULATIONS}

At the spawn level, the mean of $LengthMacroT$ is $LengthMacroMeanS$, the mean of $LengthT$ is $LengthMeanS$, and the sum of $AreaT$ is $AreaTotS$, all in metres.
In addition, the sum of $PlantsTotT$ is $PlantsTotS$, the sum of $FrondsTotT$ is $FrontsTotS$, the mean of $HeightMeanT$ is $HeightMeanS$, and the mean of $LayersMeanT$ is $LayersMeanS$. 
The number of fronds per plant is
\begin{equation}
FrondsPerPlantS = \frac{FrondsTotS} {PlantsTotS} \, .
\label{eqFrondsPerPlant}
\end{equation}
The number of eggs per plant in thousands is \citep{HaegeleSchweigert1990}
\begin{multline}
EggsPerPlantS = 0.073 \times LayersMeanS^{0.673} \times \\ 
HeightMeanS^{0.932} \times FrondsPerPlantS^{0.703} \times 1\,000
\label{eqEggsPerPlantMacro}
\end{multline}
where $EggsPerPlantS$ is in $\text{Eggs} \cdot 10^{3} \cdot \text{plant}^{-1}$. 
Macrocystis egg density in thousands is
\begin{equation}
EggDensS = \frac{EggsPerPlantS \times PlantsTotS} {AreaTotS}
\label{eqEggDensityMacro}
\end{equation}
where $EggDenS$ is in $\text{Eggs} \cdot 10^{3} \cdot \text{m}^{-2}$.
The Macrocystis spawn index is
\begin{equation}
MacroSSB = \frac{EggDensS \times LengthMacroMeanS \times LengthMeanS} {100\,000}
\label{eqBiomassMacro}
\end{equation}
where $MacroSSB$ is in (units?), based on fecundity estimates by \citet{Hay1985}.

\section{UNDERSTORY SPAWN}

Understory spawn surveys collect data in quadrats, and calculations are done at the quadrat, transect, and spawn levels.
There are two separate calculations for egg density at the quadrat level: spawn on the bottom, and spawn on vegetation (i.e., algae).

\subsection{QUADRAT LEVEL CALCULATIONS}

Bottom egg density in thousands is \citep{HaegeleEtal1979}
\begin{equation}
EggsDBot = 340 \times BotLayers \times BotProp
\label{eqEggDensUnderB}
\end{equation}
where $BotLayers$ is the number of egg layers on the bottom, $BotProp$ is the proporiton of the bottom covered in spawn, and $EggsDBot$ is in $\text{Eggs} \cdot 10^{3} \cdot \text{m}^{-2}$.
Vegetation egg density in thousands is \citep{Schweigert2005}
\begin{multline}
EggsDVeg = 600.567 \times VegLayers^{0.6355} \times VegProp^{1.413} \times V \times 1.0512
\label{eqEggDensUnderV}
\end{multline}
where $VegLayers$ is the numer of egg layers on vegetation, $VegProp$ is the proportion of the bottom covered by the vegetation type, $V$ is the vegetation coefficient (Table \ref{tabVegTypes}), and $EggsDVeg$ is in $\text{Eggs} \cdot 10^{3} \cdot \text{m}^{-2}$.
The total linear weighted understory egg density in thousands is%
\footnote{Explain why we calculate weighted egg density, $EggDensWtQ$.
Is it because transects are different lengths (which we call widths)?}
\begin{equation}
EggDensWtQ = \left( EggsDBot + EggsDVeg \right) \times Width
\label{eqEggDensWtUnder}
\end{equation}
where $Width$ is the transect width in metres,%
\footnote{$Width$ may be expanded in certain years to account for footrope expansion when wet.}
and $EggDensWtQ$ is in $\text{Eggs} \cdot 10^{3} \cdot \text{m}^{-1}$.

\begin{table}
\centering
\caption{Vegetation (i.e., algae) types and coefficients, $V$, for estimating egg density \citep{Schweigert2005}.}
\begin{tabular}{lr}
\toprule
Vegetation type & Coefficient, $V$\\
\midrule
Grasses & 0.9715 \\
Grunge & 1.0000 \\
Kelp, flat & 0.9119 \\
Kelp, standing & 1.1766 \\
Leafy algae & 0.6553 \\
Rockweed & 0.7793 \\
Sargassum & 1.1766 \\
Stringy algae & 1.0000 \\
\bottomrule
\end{tabular}
\label{tabVegTypes}
\end{table}

\subsection{TRANSECT LEVEL CALCULATIONS}

At the transect level, the mean $EggDensWtQ$ is $EggDensWtMeanT$.

\subsection{SPAWN LEVEL CALCULATIONS}

At the spawn level, the sum of transect widths $Width$ is $WidthTotS$, the mean of $Width$ is $WidthMeanS$, and the vegetation length is $VegLengthS$, all in metres.%
\footnote{$VegLength$ will be equal to the transect length if there is no recorded vegetation length.} 
Also, the sum of $EggDensWtMeanT$ is $EggDensWtTotS$.
Understory egg density is 
\begin{equation}
EggDensWtS = \frac{EggDensWtTotS} {WidthTotS}
\label{eqEggDensityUnder}
\end{equation}
where $EggDensWtS$ is in $\text{Eggs} \cdot 10^{3} \cdot \text{m}^{-2}$.
The understory spawn index is
\begin{equation}
UnderSSB = \frac{EggDensWtS \times VegLengthS \times WidthMeanS} {100\,000}
\label{eqBiomassUnder}
\end{equation}
where $UnderSSB$ is in (units?), based on fecundity estimates by \citet{Hay1985}.

\section{TOTAL SPAWN}

The total spawn index for each spawn is
\begin{equation}
TotalSSB = SurfSSB + MacroSSB + UnderSSB
\label{eqTotalSSB}
\end{equation}
where $TotalSSB$ is in (units?).

\section{ACKNOWLEDGEMENTS}

Keep in mind:
\begin{itemize}
\item Jake Schweigert
\end{itemize}

% References
\bibliographystyle{Document/CJFAS}
\addcontentsline{toc}{section}{\refname}\bibliography{Document/Grinnell}

% The end
\end{document}
